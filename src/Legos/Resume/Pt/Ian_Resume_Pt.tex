\documentclass[a4paper,12pt]{article}
\usepackage[utf8]{inputenc}
\usepackage[T1]{fontenc}
\usepackage[margin=0.7in, bottom=0.3in]{geometry} % Adjusted bottom margin
\usepackage{hyperref}
\usepackage{enumitem}
\usepackage{titlesec}
\usepackage{xcolor}
\usepackage{multicol}

\titleformat{\section}
  {\Large\bfseries\color{blue}}
  {}{0em}
  {}[\titlerule]

\pagestyle{empty}

\begin{document}

\vspace*{-2cm}

\begin{center}
\Huge\textbf{Ian De Holanda Cavalcanti Bezerra}

\LARGE Estudante de Matemática Aplicada e Ciência da Computação

\large\href{mailto:idhcb.ian@usp.br}{idhcb.ian@usp.br}
\end{center}

\section*{Perfil}
\vspace{-0.5em}
\begin{multicols}{2}
\setlength{\columnsep}{0.4cm}
\setlength{\parindent}{0pt}
\setlength{\parskip}{0.15em}
\normalsize

Sou um estudante de Matemática Aplicada e Ciência da Computação no ICMC USP, apaixonado por resolver problemas difíceis e construir o futuro. Na universidade desenvolvi grande interesse em IA e computação o que me levou a explorar o uso de sistemas mais inteligentes na expansão do conhecimento e das conquistas humanas. Acredito na eficiência e simplicidade, valorizando a execução rápida e melhorias iterativas para alcançar grandes feitos.

Desde a infância, sou fascinado por ciência e tecnologia. Já no ensino fundamental, participei de um grupo de extensão onde criamos o jogo Pong usando Scratch, no ensino médio, me aprofundei em computadores e sistemas, aprendendo Python e Linux. Reconhecendo a importância fundamental da matemática na compreensão da realidade e dos sistemas inteligentes, escolhi cursar Matemática Aplicada e Ciência da Computação na Universidade de São Paulo.

Na universidade, expandi meu conhecimento em estruturas de dados, algoritmos e sistemas de programação de baixo nível como C e Assembly. Meus projetos paralelos exploram inteligência artificial e arquiteturas modernas de Redes Neurais. Recentemente, me interessei pelos aspectos comerciais da programação, aprendendo tecnologias web como HTML/CSS/JavaScript e frameworks como Node.js e React para transformar sistemas inteligentes e algoritmos em produtos.

\end{multicols}
\vspace{-0.5em}

\section*{Educação}
\vspace{-0.5em}
\begin{itemize}[leftmargin=*, itemsep=-1.5pt]
  \item \textbf{Colégio Bandeirantes} - São Paulo
    \begin{itemize}[itemsep=-1pt]
      \item Concluído: 2021
    \end{itemize}
  \item \textbf{Bacharelado em Matemática Aplicada e Ciência da Computação}
    \begin{itemize}[itemsep=-1pt]
      \item Universidade de São Paulo (USP), São Carlos
      \item Previsão de Conclusão: 2026
    \end{itemize}
\end{itemize}
\vspace{-0.5em}

\section*{Habilidades}
\vspace{-0.5em}
\begin{itemize}[leftmargin=*, itemsep=-1.5pt]
  \item Linguagens de Programação (Proficiente): Python, C
  \item Linguagens de Programação (Básico): Assembly, Rust
  \item Desenvolvimento Web: HTML, CSS, JavaScript, React, Node.js, D3.js
  \item Análise de Dados: MATLAB, SQL
  \item Bibliotecas Python: PyTorch, Numpy, Matplotlib, Pandas, Multiprocessing
  \item Infraestrutura: AWS (EC2, Amplify, S3, APIGateway), Docker, Git
  \item Linux (Proficiente)
  \item Base matemática: Análise Numérica, Probabilidade e Otimização.
\end{itemize}
\vspace{-0.5em}

\section*{Projetos e Experiências}
\vspace{-0.5em}
\begin{itemize}[leftmargin=*, itemsep=-1.5pt]
  \item Membro do grupo de extensão de programação CodeLab
  \item Desenvolvimento de um IA em PyTorch para classificação de moléculas orgânicas
  \item Universidade de São Paulo: Projeto de Iniciação Científica sobre visualização de grafos de embeddings de Redes Neurais e aprendizado semi-supervisionado com Redes Convolucionais de Grafos.
  \item BrickLugo: Desenvolveu um modelo de precificação para Airbnb e o site da Startup (Frontend e Backend).
  \item Site pessoal com portfólio e projetos: \href{https://iansmainframe.com}{iansmainframe.com}
\end{itemize}
\vspace{-0.5em}

\section*{Idiomas}
\vspace{-0.5em}
\begin{itemize}[leftmargin=*, itemsep=-1.5pt]
  \item Português (Nativo)
  \item Inglês (Fluente)
\end{itemize}

\end{document}
